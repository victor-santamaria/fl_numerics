\documentclass[preprint,1p]{amsart}

\usepackage{amsfonts}
\usepackage{amsmath}
\usepackage{amsthm}
\usepackage[utf8]{inputenc}
\usepackage[english]{babel}
\usepackage{epstopdf}
\usepackage{bmpsize}
\usepackage{epsfig}
\usepackage{hyperref}
\usepackage[english]{varioref}
\usepackage{amssymb}
\usepackage{multicol}
\usepackage{dcolumn}
\usepackage{geometry}
\usepackage{fancyhdr}
\usepackage[mathcal]{eucal}
\usepackage{mathrsfs}
\usepackage{color, colortbl}
\usepackage{microtype}
\usepackage{longtable}
\usepackage[toc,page]{appendix}
\usepackage{bm}
\usepackage{pifont}
\usepackage{fleqn}
\usepackage{graphicx}
\usepackage{txfonts}
\usepackage{subfig}
\usepackage[pagewise]{lineno}\linenumbers
\DeclareMathAlphabet{\mathbbm}{U}{bbm}{m}{n}% from bbm.sty

\input fig4tex.tex

\allowdisplaybreaks
\DisableLigatures{encoding = *, family = * }

\numberwithin{equation}{section}
\pagestyle{myheadings}

\newtheorem{theorem}{Theorem}[section]
\newtheorem{proposition}{Proposition}[section]
\newtheorem{lemma}{Lemma}[section]
\newtheorem{definition}{Definition}[section]
\newtheorem{remark}{Remark}[section]

\numberwithin{equation}{section}
\numberwithin{theorem}{section}
\numberwithin{remark}{section}
\numberwithin{lemma}{section}
\numberwithin{proposition}{section}
\numberwithin{definition}{section}

\newcommand{\norm}[2]{{\left\|#1\right\|}_{#2}}
\newcommand{\fl}[2]{(-d_x^2)^{#1}#2}
\newcommand{\rfl}[2]{A^{#1}_{\Omega}#2}
\newcommand{\hp}[1]{\hphantom{#1}}
\newcommand{\cns}{c_{N,s}}
\newcommand{\ccs}{c_{1,s}}
\newcommand{\ffl}[2]{(-d_x^{\,2})^{#1}#2}
\newcommand{\flh}[2]{\frac{1}{\Gamma(-s)}\int_0^{+\infty}\Big(e^{t\Delta}#2 - #2\Big)\frac{dt}{t^{1+#1}}}
\newcommand{\kernel}[1]{|x-y|^{#1}}
\newcommand{\dkj}{\delta_{kj}}
\newcommand{\intr}[1]{\underset{#1}{\int}}
\newcommand{\Do}[1]{D_{#1}}
\newcommand{\Hs}{H^s_0(\Omega)}
\newcommand{\ue}[1]{#1^{\,\varepsilon}}
\newcommand{\xHdot}[1]{\dot{H}^{#1}}
\newcommand{\ha}[2]{\mathbf{H}_{#1}^{#2}}
\newcommand{\lhi}{\mathcal{L}_i^h}
\newcommand{\NN}{\mathbb{N}}
\newcommand{\ZZ}{\mathbb{Z}}
\newcommand{\RR}{\mathbb{R}}
\newcommand{\CC}{\mathbb{C}}
\newcommand{\TT}{\mathbf{T}}

\usepackage{tikz}
\usepackage{pgfplots}
\usepackage{pgfplotstable}


% The mesh
\def\mesh{\mathfrak{M}}
\def\petitmesh{{\scriptscriptstyle \mathfrak{M}}}
% \def\petitmesh{{\tiny \mathfrak{M}}}
% \def\petitmesh{{\scriptscriptstyle  M}}

% Mesh size
\def\hM{h^\petitmesh}

% The space of discrete functions
\def\fM{{\R^\petitmesh}}  % On the primal mesh 
\def\fdM{{\R^{\overline{\petitmesh}}}} % On the dual mesh
\def\fMdM{{\R^{\petitmesh\cup\partial \petitmesh}}}  % On the primal mesh + boundary
% The  discrete Laplace
\def\AM{{\mathcal{A}^\petitmesh}}
% The parabolic operators
\def\PM{{\mathcal{P}^\petitmesh_\pm}}
\def\PMp{{\mathcal{P}^\petitmesh_+}}
\def\PMm{{\mathcal{P}^\petitmesh_-}}

\def\ah{a^\petitmesh}
\def\bh{b^\petitmesh}

\def\yh{y^\petitmesh}
\def\qh{q^\petitmesh}
\def\zh{z^\petitmesh}
\def\ph{p^\petitmesh}

\def\phz{p^{\petitmesh}_0}
\def\yhz{y^{\petitmesh}_0}
\def\whz{w^{\petitmesh}_0}

\def\LtwoDT{{L^2_{\DT}(0,T;\fM)}}

\def\vh{v^\petitmesh}
\def\xih{\xi^\petitmesh}

\def\yhdM{y^{\partial \petitmesh}}
\def\qhdM{q^{\partial \petitmesh}}
\def\phdM{p^{\partial \petitmesh}}
\def\zhdM{z^{\partial \petitmesh}}

\def\DT{{\delta t}}
\def\eps{\varepsilon}

\newcommand\dsp{\displaystyle}

\newcommand\inter[1]{\llbracket #1\rrbracket}

\newcommand\rouge[1]{{\color{red} #1}}

 \usepackage{tikz}
 \usepackage{pgfplots}
 \usepackage{subfig}
 \pgfplotsset{surface/.style={ %
               xmax=0.34,%
               axis z line=center,%
               axis x line=center,%
               axis y line=center,%
               %axis on top,%
               zmin=-1,%
               clip=false,%
               extra x ticks={1},%
               extra x tick label={$T=1$},%
               xtick={0},%
               ytick=\empty}}


\newcommand{\controldomain}[2]{ \addplot3 [surf,fill=violet,mesh/rows=2] coordinates {(0,#1,0) (1,#1,0) (0,#2,0) (1,#2,0)}; }
\newcommand{\controldomainT}[3]{ \addplot3 [surf,fill=violet,mesh/rows=2] coordinates {(0,#1,0) (#3,#1,0) (0,#2,0) (#3,#2,0)}; }

\newcommand{\couplingdomain}[2]{ \addplot3 [surf,fill=green,mesh/rows=2] coordinates {(0,#1,0) (1,#1,0) (0,#2,0) (1,#2,0)}; }
\newcommand{\couplingdomainT}[3]{ \addplot3 [surf,fill=green,mesh/rows=2] coordinates {(0,#1,0) (#3,#1,0) (0,#2,0) (#3,#2,0)}; }

 \pgfplotsset{erreurs/.style={scale=1,
     legend cell align=left,
     legend pos=outer north east,
     legend plot pos=right,
     legend style={cells={anchor=east},draw=none},
     xlabel=$h$,
    xmin=0.001,xmax=0.05}}

\tikzset{pente/.style={opacity=0.6}}

\pgfplotsset{cout/.style={black,mark=diamond*,mark size=2.5,mark options={fill=gray}}}
\pgfplotsset{cible/.style={black,mark=square*,mark size=2.5,mark options={fill=gray}}}
\pgfplotsset{cibleyT/.style={black,mark=*,mark size=2.5,mark options={fill=gray}}}
\pgfplotsset{CG/.style={black,mark=otimes*,mark size=2.5,mark options={fill=gray}}}
%\pgfplotsset{solex/.style={black,mark=diamond*,mark size=2.5,mark options={fill=gray}}}
\pgfplotsset{solex/.style={black,mark=*,mark size=2.5,mark options={fill=gray}}}
\pgfplotsset{energie/.style={black,mark=triangle*,mark size=2.5,mark options={fill=gray}}}


%%% Local Variables:
%%% mode: latex
%%% TeX-master: "insensitizing_semi"
%%% End:


\title[FE approximation of the 1-d fractional Poisson equation]{Finite Elements approximation of the one-dimensional fractional Poisson equation with applications to numerical control}

\author{U.~Biccari}
\address{Umberto Biccari, DeustoTech, University of Deusto, 48007 Bilbao, Basque Country, Spain.}
\address{Umberto Biccari, Facultad Ingenier\'{\i}a, Universidad de Deusto, Avda Universidades 24, 48007 Bilbao, Basque Country, Spain.}
\email{umberto.biccari@deusto.es,u.biccari@gmail.com}
%
\author{V.~Hern\'andez-Santamar\'ia}
\address{Victor Hern\'andez-Santamar\'ia, DeustoTech, University of Deusto, 48007 Bilbao, Basque Country, Spain.}
\address{Victor Hern\'andez-Santamar\'ia, Facultad Ingenier\'{\i}a, Universidad de Deusto, Avda Universidades 24, 48007 Bilbao, Basque Country, Spain.}
\email{victor.santamaria@deusto.es}

\thanks{The work of Umberto Biccari was partially supported by the Advanced Grant DYCON (Dynamic Control) of the European Research Council Executive Agency, by the MTM2014-52347 Grant of the MINECO (Spain) and by the Air Force Office of Scientific Research under the Award No: FA9550-15-1-0027. The work of V\'ictor Hern\'andez-Santamar\'ia was partially supported by the Advanced Grant DYCON (Dynamic Control) of the European Research Council Executive Agency.}










