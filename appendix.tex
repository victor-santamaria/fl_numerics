{\appendix

\section{Explicit computations of the elements of the matrix $\mathcal A_h$}\label{appendix}

We present here the explicit computations for each element $a_{i,j}$ of the stiffness matrix, completing the discussion that we started in Section \ref{fe_sec}.

\subsubsection*{Step 1: $j\geq i+2$}
We recall that, in this case, the value of $a_{i,j}$ is given by the integral
\begin{align}\label{elem_noint_app}
	a_{i,j}=-2 \int_{x_{j-1}}^{x_{j+1}}\int_{x_{i-1}}^{x_{i+1}}\frac{\phi_i(x)\phi_j(y)}{|x-y|^{1+2s}}\,dxdy.
\end{align}

In Fig. \ref{upp_tri}, we give a scheme of the region of interaction (marked in grey) between the basis functions in this case. 
\begin{figure}[h]
\figinit{0.7pt}
% Axes
\figpt 0:(-90,-10)
\figpt 1:(-100,0) \figpt 2:(220,0)
\figpt 3:(-80,-20) \figpt 4:(-80,290)

%%%%
\figpt 5:(-50,0) \figpt 6:(-50,230)
\figpt 7:(-25,0) \figpt 8:(-25,230)
\figpt 9:(0,0) \figpt 10:(0,230)
%
\figpt 11:(-80,30) \figpt 12:(150,30)
\figpt 13:(-80,55) \figpt 14:(150,55)
\figpt 15:(-80,80) \figpt 16:(150,80)
%
\figpt 17:(100,0) \figpt 18:(100,230)
\figpt 19:(125,0) \figpt 20:(125,230)
\figpt 21:(150,0) \figpt 22:(150,230)
%
\figpt 23:(-80,180) \figpt 24:(150,180)
\figpt 25:(-80,205) \figpt 26:(150,205)
\figpt 27:(-80,230) \figpt 28:(150,230)

% Grey squares 
\figpt 29:(-50,30) \figpt 30:(0,30)
\figpt 31:(-50,80) \figpt 32:(0,80)

\figpt 33:(-50,180) \figpt 34:(0,180)
\figpt 35:(-50,230) \figpt 36:(0,230)

\figpt 37:(100,180) \figpt 38:(150,180)
\figpt 39:(100,230) \figpt 40:(150,230)

\figpt 41:(100,30) \figpt 42:(150,30)
\figpt 43:(100,80) \figpt 44:(150,80)

% Dotted crosses
\figpt 45:(-50,55) \figpt 46:(0,55)
\figpt 47:(-25,30) \figpt 48:(-25,80)

\figpt 49:(-50,205) \figpt 50:(0,205)
\figpt 51:(-25,180) \figpt 52:(-25,230)

\figpt 53:(100,205) \figpt 54:(150,205)
\figpt 55:(125,180) \figpt 56:(125,230)

\figpt 57:(100,55) \figpt 58:(150,55)
\figpt 59:(125,30) \figpt 60:(125,80)

% Points for writing
\figpt 61:(-50,-10) \figpt 62:(-25,-10) \figpt 63:(0,-10) 
\figpt 64:(-92,30) \figpt 65:(-92,55) \figpt 66:(-92,80) 
\figpt 67:(100,-10) \figpt 68:(125,-10) \figpt 69:(150,-10) 
\figpt 70:(-92,180) \figpt 71:(-92,205) \figpt 72:(-92,230) 

\figpt 73:(-92,280) \figpt 74:(210,-10)

% Diagonal

\figpt 75:(-80,0) \figpt 76:(180,260)


% 2. Creation of the graphical file
\figdrawbegin{}
\figdrawarrow[1,2]
\figdrawarrow[3,4]
\figset(dash=4)
\figdrawline[5,6]
\figdrawline[7,8]
\figdrawline[9,10]
\figdrawline[11,12]
\figdrawline[13,14]
\figdrawline[15,16]
\figdrawline[17,18]
\figdrawline[19,20]
\figdrawline[21,22]
\figdrawline[23,24]
\figdrawline[25,26]
\figdrawline[27,28]
\figset(dash=default)

\figset(fillmode=yes, color=0.8)
%\figdrawline[29,30,32,31,29]
\figdrawline[33,34,36,35,33]
%\figdrawline[37,38,40,39,37]
\figdrawline[41,42,44,43,41]
\figset (fillmode=no, color=default)
%\figdrawline[29,30,32,31,29]
\figdrawline[33,34,36,35,33]
%\figdrawline[37,38,40,39,37]
\figdrawline[41,42,44,43,41]

\figset(dash=4)
\figdrawline[45,46]
\figdrawline[47,48]
\figdrawline[49,50]
\figdrawline[51,52]
\figdrawline[53,54]
\figdrawline[55,56]
\figdrawline[57,58]
\figdrawline[59,60]

\figset(dash=5)
\figdrawline[75,76]

\figdrawend

% 3. Writing text on the figure
\figvisu{\figBoxA}{}{
\figwritec [0]{$O$}
\figwritec [61,64]{$x_{i-1}$}
\figwritec [62,65]{$x_i$}
\figwritec [63,66]{$x_{i+1}$}
\figwritec [67,70]{$x_{j-1}$}
\figwritec [68,71]{$x_j$}
\figwritec [69,72]{$x_{j+1}$}
\figwritec [73]{$y$}
\figwritec [74]{$x$}
}
\centerline{\box\figBoxA}
\caption{Interactions between the basis function $\phi_i$ and $\phi_j$ when $j\geq i+2$.}\label{upp_tri}
\end{figure}

Now, taking into account the definition of the basis function \eqref{basis_fun}, the integral \eqref{elem_noint_app} becomes
\begin{align*}
	a_{i,j}=-2 \int_{x_{j-1}}^{x_{j+1}}\int_{x_{i-1}}^{x_{i+1}}\frac{\left(1-\frac{|x-x_i|}{h}\right)\left(1-\frac{|y-x_j|}{h}\right)}{|x-y|^{1+2s}}\,dxdy.
\end{align*}

Let us introduce the following change of variables:
\begin{align}\label{cv_appendix}
	\frac{x-x_i}{h}=\hat{x},\;\;\; \frac{y-x_i}{h}=\hat{y}.
\end{align}

Then, rewriting (with some abuse of notations since there is no possibility of confusion) $\hat{x}=x$ and $\hat{y}=y$, we get 
\begin{align}\label{elem_noint_cv}
	a_{i,j}=-2h^{1-2s} \int_{-1}^1\int_{-1}^1\frac{(1-|x\,|\,)(1-|y\,|\,)}{|x-y+i-j\,|^{1+2s}}\,dxdy.
\end{align}

The integral \eqref{elem_noint_cv} can be computed explicitly in the following way. First of all, for simplifying the notation, let us define $k=j-i$. We have 
\begin{align*}
	a_{i,j} = &\, -2h^{1-2s} \int_{-1}^1\int_{-1}^1\frac{(1-|x\,|\,)(1-|y\,|\,)}{|x-y+i-j\,|^{1+2s}}\,dxdy =-2h^{1-2s} \int_{-1}^1\int_{-1}^1\frac{(1-|x\,|\,)(1-|y\,|\,)}{|x-y-k\,|^{1+2s}}\,dxdy
	\\
	= &\, -2h^{1-2s} \int_0^1\int_0^1\frac{(1-x)(1-y)}{(y-x+k)^{1+2s}}\,dxdy - 2h^{1-2s} \int_0^1\int_{-1}^0\frac{(1+x)(1-y)}{(y-x+k)^{1+2s}}\,dxdy 
	\\
	&- 2h^{1-2s} \int_{-1}^0\int_0^1\frac{(1-x)(1+y)}{(y-x+k)^{1+2s}}\,dxdy - 2h^{1-2s} \int_{-1}^0\int_{-1}^0\frac{(1+x)(1+y)}{(y-x+k)^{1+2s}}\,dxdy
	\\
	= &\, - 2h^{1-2s}(B_1 + B_2 + B_3 + B_4).
\end{align*}

These terms $B_i$, $i=1,2,3,4$, can be computed integrating by parts several times. In more detail, we have
\begin{align*}
	& B_1 = \frac{1}{4s(1-2s)}\left[2k^{1-2s}-\frac{(k+1)^{2-2s}-(k-1)^{2-2s}}{1-s}-\frac{2k^{3-2s}-(k+1)^{3-2s}-(k-1)^{3-2s}}{(1-s)(3-2s)}\right]
	\\
	& B_2 = \frac{1}{4s(1-2s)}\left[-2k^{1-2s}+\frac{2(k+1)^{2-2s}-2k^{2-2s}}{1-s}+\frac{2(k+1)^{3-2s}-k^{3-2s}-(k+2)^{3-2s}}{(1-s)(3-2s)}\right]
	\\
	& B_3 = \frac{1}{4s(1-2s)}\left[-2k^{1-2s}+\frac{2k^{2-2s}-2(k-1)^{2-2s}}{1-s}+\frac{2(k-1)^{3-2s}-k^{3-2s}-(k-2)^{3-2s}}{(1-s)(3-2s)}\right]
	\\
	& B_4 = \frac{1}{4s(1-2s)}\left[2k^{1-2s}-\frac{(k+1)^{2-2s}-(k-1)^{2-2s}}{1-s}-\frac{2k^{3-2s}-(k+1)^{3-2s}-(k-1)^{3-2s}}{(1-s)(3-2s)}\right].
\end{align*} 

Therefore, we obtain
\begin{align*}
	a_{i,j} = - h^{1-2s}\,\frac{4(k+1)^{3-2s} + 4(k-1)^{3-2s}-6k^{3-2s}-(k+2)^{3-2s}-(k-2)^{3-2s}}{2s(1-2s)(1-s)(3-2s)}.
\end{align*} 

We notice that, when $s=1/2$, both the numerator and the denominator of the expression above are zero. Hence, in this particular case, it would not be possible to introduce the value that we just encountered in our code. Nevertheless, this difficulty can be overcome noting that we can easily compute
\begin{align*}
	\lim_{s\to\frac{1}{2}} &- h^{1-2s}\,\frac{4(k+1)^{3-2s} + 4(k-1)^{3-2s}-6k^{3-2s}-(k+2)^{3-2s}-(k-2)^{3-2s}}{2s(1-2s)(1-s)(3-2s)}
	\\
	& = -4(k+1)^2\log(k+1)-4(k-1)^2\log(k-1)+6k^2\log(k)+(k+2)^2\log(k+2)+(k-2)^2\log(k-2),
\end{align*} 
if $k\neq 2$. When $k=2$, instead, since 
\begin{align*}
	\lim_{k\to 2} (k-2)^2\log(k-2) =0,
\end{align*}
the corresponding value $a_{i,j}=a_{i,i+2}$ if given by 
\begin{align*}
	a_{i,i+2} = 56\ln(2)-36\ln(3).
\end{align*}


\subsubsection*{Step 2: $j= i+1$}
This is the most cumbersome case, since it is the one with the most interactions between the basis functions (see Fig. \ref{basis_upp_dia}). According to \eqref{stiffness_nc}, and using the symmetry of the integral with respect to the bisector $y=x$, we have 
	\begin{align*}
	a_{i,i+1}= & \int_{\RR}\int_{\RR}\frac{(\phi_i(x)-\phi_i(y))(\phi_{i+1}(x)-\phi_{i+1}(y))}{|x-y|^{1+2s}}\,dxdy
	\\
	= & \int_{x_{i+1}}^{+\infty}\int_{x_{i+1}}^{+\infty} \ldots\,dxdy + 2\int_{x_{i+1}}^{+\infty}\int_{x_i}^{x_{i+1}} \ldots\,dxdy + 2\int_{x_{i+1}}^{+\infty}\int_{-\infty}^{x_i} \ldots\,dxdy 
	\\
	& + \int_{x_i}^{x_{i+1}}\int_{x_i}^{x_{i+1}} \ldots\,dxdy + 2\int_{x_i}^{x_{i+1}}\int_{-\infty}^{x_i} \ldots\,dxdy + \int_{-\infty}^{x_i}\int_{-\infty}^{x_i} \ldots\,dxdy 
	\\
	:= & Q_1 + Q_2 + Q_3 + Q_4 + Q_5 + Q_6.
\end{align*}

In Fig. \ref{upp_dia}, we give a scheme of the regions of interactions between the basis functions $\phi_i$ and $\phi_{i+1}$ enlightening the domain of integration of the $Q_i$. The regions in grey are the ones that produce a contribution to $a_{i,i+1}$, while on the regions in white the integrals will be zero.
\begin{figure}[h]
\figinit{0.7pt}
% Axes
\figpt 0:(-50,70)
\figpt 1:(-100,80) \figpt 2:(220,80)
\figpt 3:(-40,-20) \figpt 4:(-40,290)

%%%%
\figpt 5:(40,0) \figpt 6:(40,270)
\figpt 7:(65,0) \figpt 8:(65,270)
\figpt 9:(90,0) \figpt 10:(90,270)
%
\figpt 11:(-80,160) \figpt 12:(200,160)
\figpt 13:(-80,185) \figpt 14:(200,185)
\figpt 15:(-80,210) \figpt 16:(200,210)
%

% Grey part 
\figpt 29:(65,210) \figpt 30:(90,210)
\figpt 31:(65,235) \figpt 32:(40,235) 
\figpt 33:(40,210) \figpt 34:(90,185)
\figpt 35:(115,185) \figpt 36:(115,160)
\figpt 37:(90,160) \figpt 38:(65,185)
\figpt 39:(40,185) \figpt 40:(65,160) 
\figpt 41:(115,0) \figpt 42:(115,270) 
\figpt 43:(115,210) \figpt 44:(-80,235) 
\figpt 45:(90,235) \figpt 46:(200,235) 

% Diagonal 
\figpt 47:(-40,80) \figpt 48:(150,270)

% Points for writing
\figpt 49:(-50,164) \figpt 50:(-47,189) 
\figpt 51:(-50,215) \figpt 52:(-50,240) 

\figpt 53:(28,74) \figpt 54:(59,74) 
\figpt 55:(80,74) \figpt 56:(104,74) 

\figpt 73:(-52,280) \figpt 74:(210,70)

\figpt 75:(160,255) \figpt 76:(77.5,255)
\figpt 77:(160,197.5) \figpt 78:(52.5,222.5)
\figpt 79:(102.5,172.5) \figpt 80:(77.5,197.5)
\figpt 81:(77.5,28) \figpt 82:(-15,197.5)
\figpt 83:(-15,28) 


% 2. Creation of the graphical file
\figdrawbegin{}

\figset(fillmode=yes, color=0.9)
\figdrawline[29,30,10,8,29]
\figdrawline[29,31,32,33,29]
\figdrawline[30,34,14,16,30]
\figdrawline[35,36,37,34,35]
\figdrawline[34,38,7,9,34]
\figdrawline[38,29,15,13,38]
\figset(fillmode=yes, color=0.6)
\figdrawline[29,30,34,38,29]
\figset(fillmode=no, color=0)
\figdrawarrow[1,2]
\figdrawarrow[3,4]
\figdrawline[8,31]
\figdrawline[10,30]
\figdrawline[30,16]
\figdrawline[35,14]
\figdrawline[35,36]
\figdrawline[36,37]
\figdrawline[37,9]
\figdrawline[38,7]
\figdrawline[38,13]
\figdrawline[33,15]
\figdrawline[32,33]
\figdrawline[32,31]
\figset(dash=4)
\figdrawline[6,32]
\figdrawline[39,5]
\figdrawline[40,11]
\figdrawline[36,12]
\figdrawline[36,41]
\figdrawline[42,43]
\figdrawline[32,44]
\figdrawline[45,46]

\figdrawline[33,30]
\figdrawline[31,45]
\figdrawline[38,35]
\figdrawline[40,37]
\figdrawline[33,39]
\figdrawline[31,38]
\figdrawline[30,37]
\figdrawline[43,35]
\figset(dash=5)
\figdrawline[47,48]
\figdrawend

% 3. Writing text on the figure
\figvisu{\figBoxA}{}{
\figwritec [0]{$O$}
\figwritec [49,53]{$x_{i-1}$}
\figwritec [50,54]{$x_i$}
\figwritec [51,55]{$x_{i+1}$}
\figwritec [52,56]{$x_{i+2}$}
\figwritec [73]{$y$}
\figwritec [74]{$x$}
\figwritec [75]{$Q_1$}
\figwritec [76,77]{$Q_2$}
\figwritec [78,79]{$Q_3$}
\figwritec [80]{$Q_4$}
\figwritec [81,82]{$Q_5$}
\figwritec [83]{$Q_6$}
}
\centerline{\box\figBoxA}
\caption{Interactions between the basis function $\phi_i$ and $\phi_{i+1}$.}\label{upp_dia}
\end{figure}

Le us now compute the terms $Q_i$, $i=1,\ldots,6$, separately. 
\subsubsection*{Computation of $Q_1$}
Since $\phi_i = 0$ on the domain of integration we have
\begin{align*}
	Q_1 &= \int_{x_{i+1}}^{+\infty}\int_{x_{i+1}}^{+\infty} \frac{\phi_{i+1}(x)-\phi_{i+1}(y)}{|x-y|^{1+2s}}\,dxdy 
	\\
	&= \int_{x_{i+1}}^{+\infty}\int_{x_{i+1}}^{+\infty} \frac{\phi_{i+1}(x)}{|x-y|^{1+2s}}\,dxdy - \int_{x_{i+1}}^{+\infty}\int_{x_{i+1}}^{+\infty} \frac{\phi_{i+1}(y)}{|x-y|^{1+2s}}\,dxdy = 0.
\end{align*}

\subsubsection*{Computation of $Q_2$}
We have
\begin{align*}
	Q_2 &= 2\int_{x_{i+1}}^{+\infty}\int_{x_i}^{x_{i+1}} \frac{\phi_i(x)(\phi_{i+1}(x)-\phi_{i+1}(y))}{|x-y|^{1+2s}}\,dxdy. 
\end{align*}

Now, using Fubini's theorem we can exchange the order of the integrals, obtaining 
\begin{align*}
	Q_2 &= 2\int_{x_i}^{x_{i+1}}\phi_i(x)\phi_{i+1}(x)\left(\int_{x_{i+1}}^{+\infty} \frac{dy}{|x-y|^{1+2s}}\right)\,dx - 2\int_{x_{i+1}}^{x_{i+2}}\int_{x_i}^{x_{i+1}} \frac{\phi_i(x)\phi_{i+1}(y)}{|x-y|^{1+2s}}\,dxdy 
	\\
	&= \frac{1}{s}\int_{x_i}^{x_{i+1}}\frac{\phi_i(x)\phi_{i+1}(x)}{(x_{i+1}-x)^{2s}}\,dx - 2\int_{x_{i+1}}^{x_{i+2}}\int_{x_i}^{x_{i+1}} \frac{\phi_i(x)\phi_{i+1}(y)}{|x-y|^{1+2s}}\,dxdy
	\\
	&= \frac{1}{s}\int_{x_i}^{x_{i+1}}\frac{\left(1-\frac{|x-x_i|}{h}\right)\left(1-\frac{|x-x_{i+1}|}{h}\right)}{(x_{i+1}-x)^{2s}}\,dx - 2\int_{x_{i+1}}^{x_{i+2}}\int_{x_i}^{x_{i+1}} \frac{\left(1-\frac{|x-x_i|}{h}\right)\left(1-\frac{|y-x_{i+1}|}{h}\right)}{|x-y|^{1+2s}}\,dxdy:= Q_2^1 + Q_2^2.
\end{align*}

The two integrals above can be computed explicitly. Indeed, employing the change of variables
\begin{align}\label{cv2}
	\frac{x_{i+1}-x}{h}=\hat{x},
\end{align}
and then renaming $\hat{x}=x$, $R_2^1$ becomes
\begin{align*}
	Q_2^1=\frac{h^{1-2s}}{s}\int_0^1 x^{1-2s}(1-x)\,dx = \frac{h^{1-2s}}{s(2-2s)(3-2s)}.
\end{align*}

For computing $Q_2^2$, instead, we introduce the change of variables
\begin{align}\label{cv3}
	\frac{x_i-x}{h}=\hat{x},\;\;\;\frac{y-x_{i+1}}{h}=\hat{y},
\end{align}
and we obtain
\begin{align*}
	Q_2^2 = -2h^{1-2s}\int_0^1\int_0^1\frac{(1-x)(1-y)}{(y-x+1)^{1+2s}}\,dxdy = h^{1-2s}\frac{2^{1-2s}+s-2}{s(1-s)(3-2s)}.
\end{align*}

Adding the two contributions, we get the following expression for the term $R_2$
\begin{align}\label{Q2}
	Q_2 = h^{1-2s}\frac{2^{2-2s}+2s-3}{s(2-2s)(3-2s)}.
\end{align}

\subsubsection*{Computation of $Q_3$}
In this case, we simply take into account the intervals in which the basis functions are supported, so that we obtain
\begin{align*}
	Q_3 &= -2\int_{x_{i+1}}^{x_{i+2}}\int_{x_{i-1}}^{x_i} \frac{\phi_i(x)\phi_{i+1}(y)}{|x-y|^{1+2s}}\,dxdy = - 2\int_{x_{i+1}}^{x_{i+2}}\int_{x_{i-1}}^{x_i} \frac{\left(1-\frac{|x-x_i|}{h}\right)\left(1-\frac{|y-x_{i+1}|}{h}\right)}{|x-y|^{1+2s}}\,dxdy.
\end{align*}

This integral can be computed applying again \eqref{cv3}, and we get
\begin{align}\label{Q3}
	Q_3 = -2h^{1-2s}\int_0^1\int_{-1}^0 \frac{(1+x)(1-y)}{(y-x+1)^{1+2s}}\,dxdy = h^{1-2s}\frac{13-5\cdot 2^{3-2s}+3^{3-2s}+s(2^{4-2s}-14)+4s^2}{2s(1-2s)(1-s)(3-2s)},
\end{align}
if $s\neq 1/2$. If $s=1/2$, instead, we have 
\begin{align*}
	Q_3 &= -2\int_0^1\int_{-1}^0 \frac{(1+x)(1-y)}{(y-x+1)^2}\,dxdy = 1+9\ln 3-16\ln(2).
\end{align*}

Notice that this last value could have been computed directly from \eqref{Q3}, by taking the limit as $s\to 1/2$ in that expression, being this limit exactly $1+9\ln 3-16\ln(2)$.

\subsubsection*{Computation of $Q_4$}
In this case, we are in the intersection of the supports of $\phi_i$ and $\phi_{i+1}$. Therefore, we have
\begin{align*}
	Q_4 &= \int_{x_i}^{x_{i+1}}\int_{x_i}^{x_{i+1}} \frac{(\phi_i(x)-\phi_i(y))(\phi_{i+1}(x)-\phi_{i+1}(y))}{|x-y|^{1+2s}}\,dxdy. 
\end{align*}

Moreover, we notice that, this time, it is possible that $x=y$, meaning that $Q_4$ could be a singular integral. To deal with this difficulty, we will exploit the explicit definition of the basis function. We have (see also Fig. \ref{basis2})
\begin{align*}
	\phi_i(x) = 1-\frac{x-x_i}{h}, \;\;\; x\in (x_i,x_{i+1}),
	\\
	\phi_{i+1}(x) = \frac{x_{i+1}-x}{h}, \;\;\; x\in (x_i,x_{i+1}).
\end{align*}
\begin{figure}[h]
\figinit{0.8pt}
% Axes
\figpt 1:(-100,0) \figpt 2:(200,0)
\figpt 11:(-90,-10) \figpt 12:(-90,140)
%%%%
\figpt 3:(-50,0) \figpt 4:(25,100) 
\figpt 5:(-50,100) \figpt 6:(25,0)
%
% Points for writing
\figpt 7:(-50,-10) \figpt 8:(25,110) 
\figpt 9:(-50,110) \figpt 10:(25,-10)

\figpt 13:(-100,130) \figpt 14:(180,-10)
\figpt 15:(-20,82) \figpt 16:(-22,10)

% 2. Creation of the graphical file
\figdrawbegin{}
\figdrawarrow[1,2]
\figset (color=\Redrgb)
\figdrawline[3,4]
\figset (color=default)
\figdrawline[5,6]
\figset(dash=4)
\figdrawline[4,6]
\figdrawarrow[11,12]
\figdrawline[3,5]
\figdrawend

% 3. Writing text on the figure
\figvisu{\figBoxA}{}{
\figwritec [7]{$(x_i,0)$}
\figwritec [8]{$(x_{i+1},1)$}
\figwritec [9]{$(x_i,1)$}
\figwritec [10]{$(x_{i+1},0)$}
\figwritec [13]{$y$}
\figwritec [14]{$x$}
\figwritec [15]{$\phi_i(x)$}
\figwritec [16]{$\color{red}\phi_{i+1}(x)$}
}
\centerline{\box\figBoxA}
\caption{Functions $\phi_i(x)$ and $\phi_{i+1}(x)$ on the interval $(x_i,x_{i+1})$.}\label{basis2}
\end{figure}
$\newline$
Therefore, 
\begin{align*}
	(\phi_i(x)-\phi_i(y))(\phi_{i+1}(x)-\phi_{i+1}(y)) = \left(\frac{y-x}{h}\right)\left(\frac{x-y}{h}\right) = -\frac{|x-y|^2}{h^2},
\end{align*}
and the integral becomes
\begin{align}\label{Q4}
	Q_4 &= -\int_{x_i}^{x_{i+1}}\int_{x_i}^{x_{i+1}} |x-y|^{1-2s}\,dxdy = -\frac{h^{1-2s}}{(1-s)(3-2s)}. 
\end{align}

\subsubsection*{Computation of $Q_5$}
Here the procedure is analogous to the one for $Q_2$ before. Using again Fubini's theorem we have
\begin{align*}
	Q_5 &= 2\int_{x_i}^{x_{i+1}}\phi_i(y)\phi_{i+1}(y)\left(\int_{-\infty}^{x_i} \frac{dx}{|x-y|^{1+2s}}\right)dy - 2\int_{x_i}^{x_{i+1}}\int_{x_{i-1}}^{x_i} \frac{\phi_i(x)\phi_{i+1}(y)}{|x-y|^{1+2s}}\,dxdy 
	\\
	&= \frac{1}{s}\int_{x_i}^{x_{i+1}}\frac{\phi_i(y)\phi_{i+1}(y)}{(y-x_i)^{2s}}\,dy - 2\int_{x_i}^{x_{i+1}}\int_{x_{i-1}}^{x_i} \frac{\phi_i(x)\phi_{i+1}(y)}{|x-y|^{1+2s}}\,dxdy. 
\end{align*}
Applying again \eqref{cv3}, it is now easy to check that $Q_5=Q_2$.

\subsubsection*{Computation of $Q_6$}
In analogy with what we did for $Q_1$, we can show that also $Q_6=0$.

\subsubsection*{Conclusion}
The elements $a_{i,i+1}$ are now given by the sum $2Q_2+Q_3+Q_4$, according to the corresponding values that we computed. In particular, we have
\begin{align}\label{Aii1}
	a_{i,i+1} = \begin{cases}
					\displaystyle h^{1-2s}\frac{3^{3-2s}-2^{5-2s}+7}{2s(1-2s)(1-s)(3-2s)}, & \displaystyle s\neq \frac{1}{2}
					\\
					9\ln 3-16\ln 2, & \displaystyle s=\frac{1}{2}.
				\end{cases}	
\end{align}

\subsubsection*{Step 3: $j= i$}
As a last step, we fill the diagonal of the matrix $\mathcal A_h$. In this case we have
	\begin{align*}
	a_{i,i}= & \int_{\RR}\int_{\RR}\frac{(\phi_i(x)-\phi_i(y))^2}{|x-y|^{1+2s}}\,dxdy
	\\
	= & \int_{x_{i+1}}^{+\infty}\int_{x_{i+1}}^{+\infty} \ldots\,dxdy + 2\int_{x_{i+1}}^{+\infty}\int_{x_{i-1}}^{x_{i+1}} \ldots\,dxdy + \int_{x_{i+1}}^{+\infty}\int_{-\infty}^{x_{i-1}} \ldots\,dxdy 
	\\
	& + \int_{x_{i-1}}^{x_{i+1}}\int_{x_{i-1}}^{x_{i+1}} \ldots\,dxdy + 2\int_{-\infty}^{x_{i-1}}\int_{x_{i-1}}^{x_{i+1}} \ldots\,dxdy + + \int_{-\infty}^{x_{i-1}}\int_{x{i+1}}^{+\infty} \ldots\,dxdy 
	\\
	& +  \int_{-\infty}^{x_{i-1}}\int_{-\infty}^{x_{i-1}} \ldots\,dxdy := R_1 + R_2 + R_3 + R_4 + R_5 + R_6 + R_7.
\end{align*}

In Fig. \ref{upp_dia}, we give a scheme of the regions of interactions between the basis functions $\phi_i(x)$ and $\phi_i(y)$ enlightening the domain of integration of the $R_i$. The regions in grey are the ones that produce a contribution to $a_{i,i}$, while on the regions in white the integrals will be zero.
\begin{figure}
\figinit{0.7pt}
% Axes
\figpt 0:(-50,70)
\figpt 1:(-100,80) \figpt 2:(220,80)
\figpt 3:(-40,-20) \figpt 4:(-40,290)

%%%%
\figpt 5:(40,0) \figpt 6:(40,270)
\figpt 7:(65,0) \figpt 8:(65,270)
\figpt 9:(90,0) \figpt 10:(90,270)
%
\figpt 11:(-80,160) \figpt 12:(200,160)
\figpt 13:(-80,185) \figpt 14:(200,185)
\figpt 15:(-80,210) \figpt 16:(200,210)

\figpt 53:(65,250) \figpt 54:(65,250)
\figpt 55:(65,230) \figpt 56:(65,25)
\figpt 57:(65,41) \figpt 58:(200,210)

\figpt 59:(-25,185) \figpt 60:(140,185)
\figpt 61:(-5,185) \figpt 62:(160,185)
\figpt 63:(65,230) \figpt 64:(65,25)

%

% Grey part 
\figpt 29:(40,210) \figpt 30:(90,210)
\figpt 31:(40,160) \figpt 32:(90,160)

% Diagonal 
\figpt 33:(-40,80) \figpt 34:(150,270)

% Points for writing
\figpt 35:(-49,164) \figpt 36:(-47,189) \figpt 37:(-49,215) 
\figpt 38:(30,74) \figpt 39:(58,74) \figpt 40:(80,74) 

\figpt 41:(150,240) \figpt 42:(65,240) \figpt 43:(150,185) 
\figpt 44:(-15,240) \figpt 45:(52.5,172.5) \figpt 46:(77.5,172.5) 
\figpt 47:(52.5,195.5) \figpt 48:(77.5,195.5) \figpt 49:(-15,185) 
\figpt 50:(65,30) \figpt 51:(150,30) \figpt 52:(-15,30) 

\figpt 73:(-52,280) \figpt 74:(210,70)


% 2. Creation of the graphical file
\figdrawbegin{}

\figset(fillmode=yes, color=0.9)
\figdrawline[29,30,10,6,29]
\figdrawline[30,32,12,16,30]
\figdrawline[32,31,5,9,32]
\figdrawline[31,11,15,29,31]
\figset(fillmode=yes, color=0.6)
\figdrawline[29,30,32,31,29]
\figset(fillmode=no, color=0)
\figdrawarrow[1,2]
\figdrawarrow[3,4]
\figdrawline[5,6]
\figdrawline[9,10]
\figdrawline[11,12]
\figdrawline[15,16]
\figset(dash=4)
\figdrawline[7,56]
\figdrawline[57,55]
\figdrawline[53,8]
\figdrawline[13,59]
\figdrawline[60,61]
\figdrawline[62,14]
\figset(dash=5)
\figdrawline[33,34]
\figdrawend

% 3. Writing text on the figure
\figvisu{\figBoxA}{}{
\figwritec [0]{$O$}
\figwritec [35,38]{$x_{i-1}$}
\figwritec [36,39]{$x_i$}
\figwritec [37,40]{$x_{i+1}$}
\figwritec [73]{$y$}
\figwritec [41]{$R_1$}
\figwritec [42,43]{$R_2$}
\figwritec [44]{$R_3$}
\figwritec [45]{$R_4^1$}
\figwritec [46]{$R_4^2$}
\figwritec [47]{$R_4^3$}
\figwritec [48]{$R_4^4$}
\figwritec [49,50]{$R_5$}
\figwritec [51]{$R_6$}
\figwritec [52]{$R_7$}
}
\centerline{\box\figBoxA}
\caption{Interactions between the basis function $\phi_i(x)$ and $\phi_i(y)$.}\label{dia}
\end{figure}
Le us now compute the terms $R_i$, $i=1,\ldots,7$, separately. First of all, according to Fig. \ref{dia} we have that $R_1=R_3=R_6=R_7=0.$ This is due to the fact that the corresponding regions are all away from the support of the basis functions. 
\subsubsection*{Computation of $R_2$}
Since $\phi_i(y) = 0$ on the domain of integrations we have
\begin{align*}
	R_2 &= 2\int_{x_{i+1}}^{+\infty}\int_{x_{i-1}}^{x_{i+1}} \frac{\phi_i^2(x)}{|x-y|^{1+2s}}\,dxdy = 2\int_{x_{i-1}}^{x_{i+1}}\phi_i^2(x)\left(\int_{x_{i+1}}^{+\infty} \frac{dy}{|x-y|^{1+2s}}\right)\,dx = \frac{1}{s}\int_{x_{i-1}}^{x_{i+1}}\frac{\phi_i^2(x)}{(x_{i+1}-x)^{2s}}\,dxdy.
\end{align*}

This integral is computed employing \eqref{cv}, and we obtain
\begin{align}\label{R2}
	R_2 = \frac{h^{1-2s}}{s}\int_{-1}^1 \frac{(1-|x|\,)^2}{(1-x)^{2s}}\,dx = h^{1-2s}\frac{4s-6+2^{3-2s}}{s(1-2s)(1-s)(3-2s)}, 
\end{align}
if $s\neq 1/2$. If $s=1/2$, instead, we have
\begin{align*}
	R_2 = 2\int_{-1}^1 \frac{(1-|x|\,)^2}{1-x}\,dx = 2\ln 16-4.
\end{align*}

\subsubsection*{Computation of $R_4$}
In this case, we are in the intersection of the supports of $\phi_i(x)$ and $\phi_i(y)$. Therefore, we have
\begin{align*}
	R_4 &= \int_{x_{i-1}}^{x_{i+1}}\int_{x_{i-1}}^{x_{i+1}} \frac{(\phi_i(x)-\phi_i(y))^2}{|x-y|^{1+2s}}\,dxdy. 
\end{align*}

In order to compute this integral, we divide it in four components as follows:
\begin{align*}
	R_4 &= \int_{x_{i-1}}^{x_i}\int_{x_{i-1}}^{x_i} \ldots\,dxdy + \int_{x_{i-1}}^{x_i}\int_{x_i}^{x_{i+1}} \ldots\,dxdy +
	\int_{x_i}^{x_{i+1}}\int_{x_{i-1}}^{x_i} \ldots\,dxdy +  \int_{x_i}^{x_{i+1}}\int_{x_i}^{x_{i+1}} \ldots\,dxdy
	\\
	&= R_4^1 + R_4^2 + R_4^3 + R_4^4.
\end{align*}

Moreover, we notice that, due to symmetry reason, we have $R_4^2 = R_4^3$. Therefore, we can compute only one of this two terms and add its value twice when building the matrix $\mathcal A_h$. Also, notice that in these two region it cannot happen that $x=y$. On the other hand, $R_4^1$ and $R_4^4$ may be singular integrals, and we shall deal with them as we did before. 

\subsubsection*{Computation of $R_4^1$}
Using again the explicit expression of the basis functions we find 
\begin{align*}
	(\phi_i(x)-\phi_i(y))^2 = \frac{|x-y|^2}{h^2},
\end{align*}
and the integral becomes
\begin{align*}
	R_4^1 &= \int_{x_{i-1}}^{x_i}\int_{x_{i-1}}^{x_i} |x-y|^{1-2s}\,dxdy = \frac{h^{1-2s}}{(1-s)(3-2s)}. 
\end{align*}

\subsubsection*{Computation of $R_4^2$}
In this case, we simply have
\begin{align*}
	R_4^2 &= \int_{x_{i-1}}^{x_i}\int_{x_i}^{x_{i+1}} \frac{(\phi_i(x)-\phi_i(y))^2}{|x-y|^{1+2s}}\,dxdy.
\end{align*}
Employing \eqref{cv} we obtain
\begin{align*}
	R_4^2 = h^{1-2s}\int_{-1}^0\int_0^1 \frac{(x+y)^2}{(x-y)^{1+2s}}\,dxdy = h^{1-2s}\frac{2s^2-5s+4-2^{2-2s}}{s(1-2s)(1-s)(3-2s)}, 
\end{align*}
if $s\neq 1/2$. If $s=1/2$, instead, we get
\begin{align*}
	R_4^2 = \int_{-1}^0\int_0^1 \frac{(x+y)^2}{(x-y)^2}\,dxdy = 3-4\ln 2.
\end{align*}
\subsubsection*{Computation of $R_4^4$}
Also in this case we can use the explicit expression of the basis functions and the integral becomes
\begin{align*}
	R_4^4 &= \int_{x_i}^{x_{i+1}}\int_{x_i}^{x_{i+1}} |x-y|^{1-2s}\,dxdy = \frac{h^{1-2s}}{(1-s)(3-2s)}=R_4^1. 
\end{align*}
Adding the values that we just computed, we therefore obtain
\begin{align}\label{R4}
	R_4 = 2(R_4^1+R_4^2) = \begin{cases}
					\displaystyle h^{1-2s}\frac{8-8s-2^{3-2s}}{2s(1-2s)(1-s)(3-2s)}, & \displaystyle s\neq \frac{1}{2}
					\\
					8\ln 3-8\ln 2, & \displaystyle s=\frac{1}{2}.
				\end{cases}	
\end{align}

\subsubsection*{Computation of $R_5$}
Since, once again, $\phi_i(y) = 0$ on the domain of integration we have
\begin{align*}
	R_5 &= 2\int_{-\infty}^{x_{i-1}}\int_{x_{i-1}}^{x_{i+1}} \frac{\phi_i^2(x)}{|x-y|^{1+2s}}\,dxdy = 2\int_{x_{i-1}}^{x_{i+1}}\phi_i^2(x)\left(\int_{-\infty}^{x_{i-1}} \frac{dy}{|x-y|^{1+2s}}\right)\,dx = \frac{1}{s}\int_{x_{i-1}}^{x_{i+1}}\frac{\phi_i^2(x)}{(x-x_{i-1})^{2s}}\,dxdy.
\end{align*}
Employing one last time \eqref{cv}, we get
\begin{align}\label{R5}
	R_5 = \frac{h^{1-2s}}{s}\int_{-1}^1 \frac{(1-|x|\,)^2}{(1+x)^{2s}}\,dx = h^{1-2s}\frac{4s-6+2^{3-2s}}{s(1-2s)(3-2s)(1-s)}=R_2,
\end{align}
if $s\neq 1/2$. If $s=1/2$, instead, we have
\begin{align*}
	R_5 = 2\int_{-1}^1 \frac{(1-|x|\,)^2}{1+x}\,dx = 8\ln 2-4.
\end{align*}

\subsubsection*{Conclusion}
The elements $a_{i,i}$ are now given by the sum $2R_2+R_4$, according to the corresponding values that we computed. In particular, we have 
\begin{align*}
	a_{i,i} = \begin{cases}
			\displaystyle h^{1-2s}\,\frac{2^{3-2s}-4}{s(1-2s)(1-s)(3-2s)}, & \displaystyle s\neq\frac{1}{2}
			\\
			\\
			8\ln 2, & \displaystyle s=\frac{1}{2}.			
			\end{cases}	
\end{align*}

}