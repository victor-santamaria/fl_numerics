\section{Numerical results} 

As a test of the efficiency of our method, we will apply it for finding the solution of
\begin{align*}
	\left\{\begin{array}{ll}
		\fl{s}{u} = 1, & x\in(-L,L)
		\\
		u\equiv 0, & x\in\RR\setminus(-L,L)
	\end{array}\right.
\end{align*}
which can also be computed analytically and it is given by (\cite{getoor1961first})
\begin{align}\label{real_sol}
	u(x)=\frac{2^{-2s}\sqrt{\pi}}{\Gamma\left(\frac{1+2s}{2}\right)\Gamma(1+s)}\Big(L^2-x^2\Big)^s.
\end{align}

In the figures below, we plotted the real solution \eqref{real_sol} of the problem considered (in red), and the numerical solution computed through the algorithm that we described (in blue). We considered $L=1$, $N=50$ and different values of $s\in(0,1)$.

\begin{figure}[!h]
	\pgfplotstableread{./figures/fl_01_50_sym.txt}{\datpu}
	\pgfplotstableread{./figures/fl_04_50_sym.txt}{\ddatpu}
		\subfloat[$s=0.1$]{
		\begin{tikzpicture}[scale=0.9]
		\begin{axis}[xmin=-1.1, xmax=1.1,legend style={at={(0.77,0.25)}}]
			\addplot [color=blue, mark=none] table[x=0,y=1]{\datpu};	
			\addplot [color=red, mark=x, only marks] table[x=2,y=3]{\datpu};		
			\addlegendentry{Numerical solution}
			\addlegendentry{Real solution}
		\end{axis}
	\end{tikzpicture}
	}
		\hspace{1cm}\subfloat[$s=0.4$]{
		\begin{tikzpicture}[scale=0.9]
		\begin{axis}[xmin=-1.1, xmax=1.1,legend style={at={(0.77,0.25)}}]
			\addplot [color=blue, mark=none] table[x=0,y=1]{\ddatpu};	
			\addplot [color=red, mark=x, only marks] table[x=2,y=3]{\ddatpu};		
			\addlegendentry{Numerical solution}
			\addlegendentry{Real solution}
		\end{axis}
	\end{tikzpicture}
	}
	\caption{Plot for different values of $s<1/2$}
\end{figure}

\begin{figure}[!h]
	\pgfplotstableread{./figures/fl_05_50_sym.txt}{\datpu}
	\pgfplotstableread{./figures/fl_08_50_sym.txt}{\ddatpu}
		\subfloat[$s=0.5$]{
		\begin{tikzpicture}[scale=0.9]
		\begin{axis}[xmin=-1.1, xmax=1.1,legend style={at={(0.77,0.25)}}]
			\addplot [color=blue, mark=none] table[x=0,y=1]{\datpu};	
			\addplot [color=red, mark=x, only marks] table[x=2,y=3]{\datpu};		
			\addlegendentry{Numerical solution}
			\addlegendentry{Real solution}
		\end{axis}
	\end{tikzpicture}
	}
		\hspace{1cm}\subfloat[$s=0.8$]{
		\begin{tikzpicture}[scale=0.9]
		\begin{axis}[xmin=-1.1, xmax=1.1,legend style={at={(0.77,0.25)}}]
			\addplot [color=blue, mark=none] table[x=0,y=1]{\ddatpu};	
			\addplot [color=red, mark=x, only marks] table[x=2,y=3]{\ddatpu};		
			\addlegendentry{Numerical solution}
			\addlegendentry{Real solution}
		\end{axis}
	\end{tikzpicture}
	}
	\caption{Plot for different values of $s\geq 1/2$}
\end{figure}

\subsection{Numerical experiments}

\begin{figure}
  \centering
\begin{tikzpicture}
  \begin{loglogaxis}[erreurs, ymin=5e-4,ymax=0.2e1,title={$s=0.8$}]
 
    \pgfplotstableread[ignore chars={|},skip first n=2]{data_paper/heat_frac_s=08.org}\resultats
    %% Original file  resultats_09-06-2017_11h18.org

    \addplot[cout] table[x=dx,y=Nv] \resultats;
    \addlegendentry{Cost of the control};
    \addplot[cible] table[x=dx,y=NyT] \resultats;
    \addlegendentry{Size of $y^M$};
    \addplot[energie] table[x=dx,y=Inf_eps(F_eps)] \resultats;
    \addlegendentry{Optimal energy};
    \draw [pente]  (axis cs: 0.003,84e-4) -- ++ (axis cs: 1, 10^0.5) -- ++ (axis cs: 10, 1) -- cycle;
    \node at (axis cs:0.003,20e-3) [right,pente] {\small slope $0.5$};
    
  \end{loglogaxis}
  
\end{tikzpicture}
\caption{Convergence properties of the method for controllability of the fractional heat equation. }\label{figure_case1}
\end{figure}

 \begin{figure}
 \centering
 \begin{tikzpicture}
  \begin{loglogaxis}[erreurs, ymin=4e-2,ymax=40.4e1,title={$s=0.4$}]
 
    \pgfplotstableread[ignore chars={|},skip first n=2]{data_paper/heat_frac_s=04.org}\resultats
    %% Original file  resultats_09-06-2017_11h47.org

    \addplot[cout] table[x=dx,y=Nv] \resultats;
    %\addlegendentry{Cost of the control};
    \addplot[cible] table[x=dx,y=NyT] \resultats;
    %\addlegendentry{Size of $y^M$};
    \addplot[energie] table[x=dx,y=Inf_eps(F_eps)] \resultats;
    %\addlegendentry{Optimal energy};
    \draw [pente]  (axis cs: 0.003,1.2e-1) -- ++ (axis cs: 1, 10^0.35) -- ++ (axis cs: 10, 1) -- cycle;
    \node at (axis cs:0.003,20e-2) [right,pente] {\small slope $0.35$};
    
    \draw [pente]  (axis cs: 0.003,80.4e0) -- ++ (axis cs: 1, {10^(-0.4)}) -- ++ (axis cs: 10, 1) -- cycle;
    \draw [pente]  (axis cs: 0.003,56.7e0) -- ++ (axis cs: 1, {10^(-0.25)}) -- ++ (axis cs: 10, 1) -- cycle;
    \node at (axis cs:0.003,23e0) [right,pente] {\small sl. $-0.25/-0.4$};
    
  \end{loglogaxis}
  
\end{tikzpicture}
\caption{Convergence properties of the method for controllability of the fractional heat equation. }\label{figure_case1}
\end{figure}

 \begin{figure}
 \centering
 \begin{tikzpicture}
  \begin{loglogaxis}[erreurs, ymin=3e-2,ymax=10e1,title={$s=0.5$}]
 
    \pgfplotstableread[ignore chars={|},skip first n=2]{data_paper/heat_frac_s=05.org}\resultats
    %% Original file  resultats_09-06-2017_11h07.org

    \addplot[cout] table[x=dx,y=Nv] \resultats;
    %\addlegendentry{Cost of the control};
    \addplot[cible] table[x=dx,y=NyT] \resultats;
    %\addlegendentry{Size of $y^M$};
    \addplot[energie] table[x=dx,y=Inf_eps(F_eps)] \resultats;
    %\addlegendentry{Optimal energy};
    \draw [pente]  (axis cs: 0.003,10e-2) -- ++ (axis cs: 1, 10^0.4) -- ++ (axis cs: 10, 1) -- cycle;
    \node at (axis cs:0.003,18e-2) [right,pente] {\small slope $0.4$};
    
    \draw [pente]  (axis cs: 0.003,2.65e1) -- ++ (axis cs: 1, {10^(-0.18)}) -- ++ (axis cs: 10, 1) -- cycle;
    \draw [pente]  (axis cs: 0.003,3.5e1) -- ++ (axis cs: 1, {10^(-0.3)}) -- ++ (axis cs: 10, 1) -- cycle;
    \node at (axis cs:0.003,1.35e1) [right,pente] {\small sl. $-0.18/-0.3$};
    
  \end{loglogaxis}
  
\end{tikzpicture}
\caption{Convergence properties of the method for controllability of the fractional heat equation. }\label{figure_case1}
\end{figure}