\section{Finite element approximation}\label{fe_sec}

\subsection{Elliptic problem}

In order to solve numerically \eqref{PE}, we will develop a finite element scheme on a uniform mesh. To this purpose, let us firstly recall the variational formulation associated to our problem: find $u\in H_0^s(-L,L)$ such that
\begin{align}\label{WF}
	\frac{\ccs}{2} \int_{\RR}\int_{\RR}\frac{(u(x)-u(y))(v(x)-v(y))}{|x-y|^{1+2s}}\,dxdy = \int_{-L}^L fv\,dx,	
\end{align}
for all $v\in H_0^s(-L,L)$. 

Now, let us now introduce a partition of the interval $(-L,L)$ as follows:
\begin{align*}
	-L = x_0<x_1<\ldots <x_i<x_{i+1}<\ldots<x_{N+1}=L\,,
\end{align*}
with $x_{i+1}=x_i+h$, $i=0,\ldots N$. We call $\mathfrak{M}$ the mesh composed by the points $\{x_i\,:\, i=1,\ldots,N\}$, while the set of the boundary points is denoted $\partial\mathfrak{M}:=\{x_0,x_{N+1}\}$. Define $K_i:=[x_i,x_{i+1}]$ and consider the discrete space 
\begin{align*}
	V_h :=\Big\{v\in H_0^s(-L,L)\,\big|\, \left. v\,\right|_{K_i}\in \mathcal{P}^1\Big\},
\end{align*} 
where $\mathcal{P}^1$ is the space of the continuous and piecewise linear functions. Hence, we approximate \eqref{WF} with the following discrete problem: find $u_h\in V_h$ such that
\begin{align}\label{WFD}
	\frac{\ccs}{2} \int_{\RR}\int_{\RR}\frac{(u_h(x)-u_h(y))(v_h(x)-v_h(y))}{|x-y|^{1+2s}}\,dxdy = \int_{-L}^L fv_h\,dx,	
\end{align}
for all $v_h\in V_h$. If now we indicate with $\big\{\phi_i\big\}_{i=1}^N$ a basis of $V_h$, it will be sufficient that \eqref{WFD} is satisfied for all the functions of the basis, since any element of $V_h$ is a linear combination of them. We therefore rewrite 
\begin{align*}
	u_h(x) = \sum_{i=1}^N u_i\phi_i(x),\;\;\; v_h(x) = \sum_{i=1}^N v_i\phi_i(x),
\end{align*} 
and \eqref{WFD} is reduced to solve the linear system $\mathcal A_h u=F$, where the stiffness matrix $\mathcal A_h\in \RR^{N\times N}$ has components
\begin{align}\label{stiffness}
	a_{i,j}=\frac{\ccs}{2} \int_{\RR}\int_{\RR}\frac{(\phi_i(x)-\phi_i(y))(\phi_j(x)-\phi_j(y))}{|x-y|^{1+2s}}\,dxdy,	
\end{align}
while the vector $F\in\RR^N$ is given by $F=(F_1,\ldots,F_N)$ with
\begin{align*}
	F_i = \int_{-L}^L f\phi_i\,dx,\;\;\; i=1,\ldots,N.
\end{align*}

Moreover, the basis $\big\{\phi_i\big\}_{i=1}^N$ that we will employ is the classical one in which each $\phi_i$ is the tent function such that $supp(\phi_i)=(x_{i-1},x_{i+1})$ and $\phi_i(x_j)=\delta_{i,j}$. In particular, for $x\in\{x_{i-1},x_i,x_{i+1}\}$ the $i^{th}$ function of the basis is explicitly defined as (see Figure \ref{basis}) 
\begin{align}\label{basis_fun}
	\phi_i(x)= 1-\frac{|x-x_i|}{h}.
\end{align} 

\begin{figure}[h]
\figinit{0.8pt}
% Axes
\figpt 1:(-100,0) \figpt 2:(200,0)
\figpt 11:(-90,-10) \figpt 12:(-90,140)
%%%%
\figpt 3:(-50,0) \figpt 4:(0,100) 
\figpt 5:(50,0) \figpt 6:(0,0)
%
% Points for writing
\figpt 7:(-50,-10) \figpt 8:(0,110) 
\figpt 9:(50,-10) \figpt 10:(0,-10)

\figpt 13:(-100,130) \figpt 14:(180,-10)
\figpt 15:(50,50)

% 2. Creation of the graphical file
\figdrawbegin{}
\figdrawarrow[1,2]
\figdrawline[3,4]
\figdrawline[4,5]
\figset(dash=4)
\figdrawline[4,6]
\figdrawarrow[11,12]

\figdrawend

% 3. Writing text on the figure
\figvisu{\figBoxA}{}{
\figwritec [7]{$(x_{i-1},0)$}
\figwritec [8]{$(x_i,1)$}
\figwritec [9]{$(x_{i+1},0)$}
\figwritec [10]{$(x_i,0)$}
\figwritec [13]{$y$}
\figwritec [14]{$x$}
\figwritec [15]{$\phi_i(x)$}
}
\centerline{\box\figBoxA}
\caption{Basis function $\phi_i(x)$ on its support $(x_{i-1},x_{i+1})$.}\label{basis}
\end{figure}
\noindent Let us now describe our algorithm. Before that, we shall make the following preliminary comments.
\begin{remark}\label{rem_prel}
The following fact are worth noticing.
\begin{itemize}
	\item It is evident from the definition \eqref{stiffness} that $\mathcal A_h$ is symmetric. Therefore, in our algorithm we will only need to compute the values $a_{i,j}$ with $j\geq i$.
	
	\item Due to the non-local nature of the problem, the matrix $\mathcal A_h$ is full. However, while computing its components we will encounter many simplifications, due to the fact that $supp(\phi_i)\cap supp(\phi_j) =\emptyset$ for $j\geq i+2$.
	
	\item With some abuse of notation, let us define 
	\begin{align}\label{stiffness_nc}
	a_{i,j}= \int_{\RR}\int_{\RR}\frac{(\phi_i(x)-\phi_i(y))(\phi_j(x)-\phi_j(y))}{|x-y|^{1+2s}}\,dxdy.
\end{align}
	In other words, while calculating the components $a_{i,j}$ we will not consider the multiplicative constant $\ccs/2$, since it is uniform for all the elements of the matrix. 
	
	\item While computing the values $a_{i,j}$, we will only work on the mesh $\mathfrak{M}$, not considering the points of the set $\partial\mathfrak{M}$. In this way, we will ensure that the basis functions $\phi_i$ satisfy the zero Dirichlet boundary conditions. In other words, in our FE approximation we are considering only the functions from $\phi_1$ to $\phi_N$. Instead, if we considered the points $x_0$ and $x_{N+1}$, then we would need to introduce in our discretization also the basis functions $\phi_0$ and $\phi_{N+1}$, which take value $1$ at the boundary, and this would not be consistent with the continuous problem. Figure \ref{basis_int} provides a graphical explanation of this last discussion.      	
\end{itemize}
\end{remark}

\begin{figure}[h]
\figinit{0.8pt}
% Axes
\figpt 1:(-110,0) \figpt 2:(420,0)
\figpt 11:(-100,-10) \figpt 12:(-100,140)
%%%%
\figpt 3:(-75,0) \figpt 4:(-25,100) 
\figpt 5:(25,0) \figpt 6:(-25,0)

\figpt 31:(-25,0) \figpt 41:(25,100) 
\figpt 51:(75,0) \figpt 61:(25,0)

\figpt 32:(25,0) \figpt 42:(75,100) 
\figpt 52:(125,0) \figpt 62:(75,0)

\figpt 33:(265,0) \figpt 43:(315,100) 
\figpt 53:(365,0) \figpt 63:(315,0)

\figpt 200:(-100,100)
%
% Points for writing
\figpt 7:(-75,-10) \figpt 8:(-25,110) 
\figpt 9:(25,-10) \figpt 10:(-25,-10)

\figpt 71:(-25,-10) \figpt 81:(25,110) 
\figpt 91:(75,-10) \figpt 101:(25,-10)

\figpt 72:(25,-10) \figpt 82:(75,110) 
\figpt 92:(125,-10) \figpt 102:(75,-10)

\figpt 73:(265,-10) \figpt 83:(315,110) 
\figpt 93:(365,-10) \figpt 103:(315,-10)

\figpt 13:(-110,130) \figpt 14:(405,-10)
\figpt 15:(50,50) \figpt 16:(250,50)

\figpt 201:(-110,100) \figpt 202:(200,50)

% 2. Creation of the graphical file
\figdrawbegin{}
\figdrawarrow[1,2]
\figdrawline[3,4]
\figdrawline[4,5]
\figset (color=\Redrgb)
\figdrawline[31,41]
\figdrawline[41,51]
\figset (color=default)
\figset (color=\ForestGreenrgb)
\figdrawline[32,42]
\figdrawline[42,52]
\figset (color=default)
\figset (color=\Bluergb)
\figdrawline[33,43]
\figdrawline[43,53]
\figset (color=default)
\figset(dash=4)
\figdrawline[4,6]
\figdrawline[200,4]
\figdrawline[41,61]
\figdrawline[42,62]
\figdrawline[43,63]
\figdrawarrow[11,12]

\figdrawend

% 3. Writing text on the figure
\figvisu{\figBoxA}{}{
\figwritec [7]{$x_0=-L$}
\figwritec [8]{$\phi_1$}
\figwritec [9]{$x_2$}
\figwritec [10]{$x_1$}
\figwritec [81]{$\phi_2$}
\figwritec [91]{$x_3$}
\figwritec [82]{$\phi_3$}
\figwritec [92]{$x_4$}
\figwritec [73]{$x_{N-1}$}
\figwritec [83]{$\phi_N$}
\figwritec [93]{$x_{N+1}=L$}
\figwritec [103]{$x_N$}
\figwritec [13]{$y$}
\figwritec [14]{$x$}
\figwritec [201]{$1$}
\figwritec [202]{$\ldots\ldots\ldots$}
\figwritec[3,53]{$\bullet$}
%\figwritec [7,93]
%\figwritec [15]{$\phi_i(x)$}
%\figwritec [16]{$\color{red}\phi_j(x)$}
}
\centerline{\box\figBoxA}
\caption{Basis functions $\phi_i(x)$ on the whole interval $(-L,L)$.}\label{basis_int}
\end{figure}

We now start building the stiffness matrix $\mathcal A_h$. This will be done it in three steps, since the values of the matrix can be computed differentiating among three well defined regions: the upper triangle, corresponding to $j\geq i+2$, the upper diagonal corresponding to $j=i+1$ and the diagonal, corresponding to $j=i$ (see Figure \ref{matrix_fig}). In fact, as it will be clear during our computations, in each of these regions the intersections among the support of the basis functions are different, thus generating different values of the bilinear form. In what follows, we will briefly present which will be the contributions to the matrix in each of these three steps, including the complete computations as an appendix at the end of the paper. 

\begin{figure}[!h]
\centering
\begin{tikzpicture}
 \matrix[matrix of math nodes,left delimiter = (,right delimiter = ),row sep=10pt,column sep = 10pt] (m)
 {
 a_{1,1}  & a_{1,2} & a_{1,3} & \ldots\ldots & \ldots\ldots & a_{1,N}\\
   &a_{2,2} & a_{2,3} & a_{2,4} & \ldots\ldots &a_{2,N}\\
   & & \ddots & \ddots & \ddots &\vdots\\
   & & & \ddots & \ddots &a_{N-2,N}\\	   
   & & & & \hspace{-0.25cm}a_{N-1,N-1} &a_{N-1,N}\\
   & & & & & a_{N,N}\\
 };
\draw[color=red] (m-1-3.north west) -- (m-1-6.north east) -- +(0.17,0) -- (m-4-6.south east) -- (m-4-6.south west) -- (m-1-3.west) -- (m-1-3.north west);
\draw[color=blue] (m-1-2.north west) -- +(0.6,0) -- +(5.37,-3.96) -- (m-5-6.north east) -- (m-5-6.south east) -- (m-5-6.south west) -- (m-1-2.south) -- (m-1-2.south west) -- (m-1-2.north west); 
\draw[color=black] (m-1-1.north west) -- (m-1-1.north east) -- +(0.2,0) -- +(0.2,-0.75) -- +(0.9,-0.75) -- +(5.75,-4.8) -- +(6.55,-4.8) -- +(6.55,-5.3) -- +(4.65,-5.3) -- (m-1-1.south west) -- (m-1-1.north west); 
\end{tikzpicture}
\caption{Structure of the stiffness matrix $\mathcal{A}_h$.}\label{matrix_fig}
\end{figure}

\subsubsection*{Step 1: $j\geq i+2$}
As we mentioned in Remark \ref{rem_prel}, in this case we have $supp(\phi_i)\cap supp(\phi_j) =\emptyset$ (see also Figure \ref{basis_upp_tri}). Hence, \eqref{stiffness_nc} is reduced to computing only the integral
\begin{align}\label{elem_noint}
	a_{i,j}=-2 \int_{x_{j-1}}^{x_{j+1}}\int_{x_{i-1}}^{x_{i+1}}\frac{\phi_i(x)\phi_j(y)}{|x-y|^{1+2s}}\,dxdy.
\end{align}

\begin{figure}[h]
\figinit{0.8pt}
% Axes
\figpt 1:(-100,0) \figpt 2:(320,0)
\figpt 11:(-90,-10) \figpt 12:(-90,140)
%%%%
\figpt 3:(-50,0) \figpt 4:(0,100) 
\figpt 5:(50,0) \figpt 6:(0,0)

\figpt 31:(150,0) \figpt 41:(200,100) 
\figpt 51:(250,0) \figpt 61:(200,0)
%
% Points for writing
\figpt 7:(-50,-10) \figpt 8:(0,110) 
\figpt 9:(50,-10) \figpt 10:(0,-10)

\figpt 71:(150,-10) \figpt 81:(200,110) 
\figpt 91:(250,-10) \figpt 101:(200,-10)

\figpt 13:(-100,130) \figpt 14:(300,-10)
\figpt 15:(50,50) \figpt 16:(250,50)

% 2. Creation of the graphical file
\figdrawbegin{}
\figdrawarrow[1,2]
\figdrawline[3,4]
\figdrawline[4,5]
\figset (color=\Redrgb)
\figdrawline[31,41]
\figdrawline[41,51]
\figset (color=default)
\figset(dash=4)
\figdrawline[4,6]
\figdrawline[41,61]
\figdrawarrow[11,12]

\figdrawend

% 3. Writing text on the figure
\figvisu{\figBoxA}{}{
\figwritec [7]{$(x_{i-1},0)$}
\figwritec [8]{$(x_i,1)$}
\figwritec [9]{$(x_{i+1},0)$}
\figwritec [10]{$(x_i,0)$}
\figwritec [71]{$(x_{j-1},0)$}
\figwritec [81]{$(x_j,1)$}
\figwritec [91]{$(x_{j+1},0)$}
\figwritec [101]{$(x_j,0)$}
\figwritec [13]{$y$}
\figwritec [14]{$x$}
\figwritec [15]{$\phi_i(x)$}
\figwritec [16]{$\color{red}\phi_j(x)$}
}
\centerline{\box\figBoxA}
\caption{Basis functions $\phi_i(x)$ and $\phi_j(x)$ for $j\geq i+1$. In this case, the supports are disjoint.}\label{basis_upp_tri}
\end{figure}

Taking into account the definition of the basis function \eqref{basis_fun}, from \eqref{elem_noint} we obtain
\begin{align*}
	a_{i,j}=-2 \int_{x_{j-1}}^{x_{j+1}}\int_{x_{i-1}}^{x_{i+1}}\frac{\left(1-\frac{|x-x_i|}{h}\right)\left(1-\frac{|y-x_j|}{h}\right)}{|x-y|^{1+2s}}\,dxdy.
\end{align*}

Finally, this last integral can be computed explicitly employing the following  change of variables:
\begin{align}\label{cv}
	\frac{x-x_i}{h}=\hat{x},\;\;\; \frac{y-x_i}{h}=\hat{y}.
\end{align}
In this way, for the elements $a_{i,j}$, $j\geq i+2$, we get the following values: 
\begin{align*}
	a_{i,j} = \begin{cases}
			\displaystyle -h^{1-2s}\,\frac{4(j-i+1)^{3-2s} + 4(j-i-1)^{3-2s}-6(j-i)^{3-2s}-(j-i+2)^{3-2s}-(j-i-2)^{3-2s}}{2s(1-2s)(1-s)(3-2s)}, & \displaystyle s\neq\frac{1}{2} 
			\\
			\\
			-4(j-i+1)^2\log(j-i+1)-4(j-i-1)^2\log(j-i-1) &  \displaystyle s=\frac{1}{2}
			\\
			\;\;\;+6(j-i)^2\log(j-i)+(j-i+2)^2\log(j-i+2)+(j-i-2)^2\log(j-i-2), &  \displaystyle j> i+2
			\\
			\\
			56\ln(2)-36\ln(3), & \displaystyle s=\frac{1}{2}
			\\
			& \displaystyle j= i+2.
		\end{cases}	
\end{align*}

\subsubsection*{Step 2: $j= i+1$}
This is the most cumbersome case, since it is the one with the most interactions between the basis functions (see Figure \ref{basis_upp_dia}). 

\begin{figure}[h]
\figinit{0.8pt}
% Axes
\figpt 1:(-100,0) \figpt 2:(200,0)
\figpt 11:(-90,-10) \figpt 12:(-90,140)
%%%%
\figpt 3:(-50,0) \figpt 4:(0,100) 
\figpt 5:(50,0) \figpt 6:(0,0)

\figpt 31:(0,0) \figpt 41:(50,100) 
\figpt 51:(100,0) \figpt 61:(50,0)
%
% Points for writing
\figpt 7:(-50,-10) \figpt 8:(0,110) 
\figpt 9:(50,-10) \figpt 10:(0,-10)

\figpt 71:(0,-25) \figpt 81:(50,110) 
\figpt 91:(100,-25) \figpt 101:(50,-25)

\figpt 13:(-100,130) \figpt 14:(180,-10)
\figpt 15:(-50,50) \figpt 16:(100,50)

% 2. Creation of the graphical file
\figdrawbegin{}
\figdrawarrow[1,2]
\figdrawline[3,4]
\figdrawline[4,5]
\figset (color=\Redrgb)
\figdrawline[31,41]
\figdrawline[41,51]
\figset (color=default)
\figset(dash=4)
\figdrawline[4,6]
\figdrawline[41,61]
\figdrawarrow[11,12]

\figdrawend

% 3. Writing text on the figure
\figvisu{\figBoxA}{}{
\figwritec [7]{$(x_{i-1},0)$}
\figwritec [8]{$(x_i,1)$}
\figwritec [9]{$(x_{i+1},0)$}
\figwritec [10]{$(x_i,0)$}
\figwritec [71]{$(x_{j-1},0)$}
\figwritec [81]{$(x_j,1)$}
\figwritec [91]{$(x_{j+1},0)$}
\figwritec [101]{$(x_j,0)$}
\figwritec [13]{$y$}
\figwritec [14]{$x$}
\figwritec [15]{$\phi_i(x)$}
\figwritec [16]{$\color{red}\phi_j(x)$}
}
\centerline{\box\figBoxA}
\caption{Basis functions $\phi_i(x)$ and $\phi_{i+1}(x)$. In this case, the intersection of the supports is the interval $[x_i,x_{i+1}]$.}\label{basis_upp_dia}
\end{figure}

According to \eqref{stiffness_nc}, and using the symmetry of the integral with respect to the bisector $y=x$, we have 
	\begin{align*}
	a_{i,i+1}= & \int_{\RR}\int_{\RR}\frac{(\phi_i(x)-\phi_i(y))(\phi_{i+1}(x)-\phi_{i+1}(y))}{|x-y|^{1+2s}}\,dxdy
	\\
	= & \int_{x_{i+1}}^{+\infty}\int_{x_{i+1}}^{+\infty} \ldots\,dxdy + 2\int_{x_{i+1}}^{+\infty}\int_{x_i}^{x_{i+1}} \ldots\,dxdy + 2\int_{x_{i+1}}^{+\infty}\int_{-\infty}^{x_i} \ldots\,dxdy 
	\\
	& + \int_{x_i}^{x_{i+1}}\int_{x_i}^{x_{i+1}} \ldots\,dxdy + 2\int_{x_i}^{x_{i+1}}\int_{-\infty}^{x_i} \ldots\,dxdy + \int_{-\infty}^{x_i}\int_{-\infty}^{x_i} \ldots\,dxdy 
	\\
	:= & Q_1 + Q_2 + Q_3 + Q_4 + Q_5 + Q_6.
\end{align*}

These contributions will be calculated separately, employing changes of variables analogous to \eqref{cv}. After several computations, we obtain
\begin{align}\label{Aii1}
	a_{i,i+1} = \begin{cases}
					\displaystyle h^{1-2s}\frac{3^{3-2s}-2^{5-2s}+7}{2s(1-2s)(1-s)(3-2s)}, & \displaystyle s\neq \frac{1}{2}
					\\
					9\ln 3-16\ln 2, & \displaystyle s=\frac{1}{2}.
				\end{cases}	
\end{align}

\subsubsection*{Step 3: $j= i$}
As a last step, we fill the diagonal of the matrix $\mathcal A_h$. In this case we have
	\begin{align*}
	a_{i,i}= & \int_{\RR}\int_{\RR}\frac{(\phi_i(x)-\phi_i(y))^2}{|x-y|^{1+2s}}\,dxdy
	\\
	= & \int_{x_{i+1}}^{+\infty}\int_{x_{i+1}}^{+\infty} \ldots\,dxdy + 2\int_{x_{i+1}}^{+\infty}\int_{x_{i-1}}^{x_{i+1}} \ldots\,dxdy + \int_{x_{i+1}}^{+\infty}\int_{-\infty}^{x_{i-1}} \ldots\,dxdy 
	\\
	& + \int_{x_{i-1}}^{x_{i+1}}\int_{x_{i-1}}^{x_{i+1}} \ldots\,dxdy + 2\int_{-\infty}^{x_{i-1}}\int_{x_{i-1}}^{x_{i+1}} \ldots\,dxdy + + \int_{-\infty}^{x_{i-1}}\int_{x{i+1}}^{+\infty} \ldots\,dxdy 
	\\
	& +  \int_{-\infty}^{x_{i-1}}\int_{-\infty}^{x_{i-1}} \ldots\,dxdy := R_1 + R_2 + R_3 + R_4 + R_5 + R_6 + R_7.
\end{align*}

\begin{figure}[h]
\figinit{0.8pt}
% Axes
\figpt 1:(-100,0) \figpt 2:(200,0)
\figpt 11:(-90,-10) \figpt 12:(-90,140)
%%%%
\figpt 3:(-50,0) \figpt 4:(0,100) 
\figpt 5:(50,0) \figpt 6:(0,0)

\figpt 31:(-40,20) \figpt 41:(-30,40) 
\figpt 51:(-20,60) \figpt 61:(-10,80)

\figpt 32:(40,20) \figpt 42:(30,40) 
\figpt 52:(20,60) \figpt 62:(10,80)
%
% Points for writing
\figpt 7:(-50,-10) \figpt 8:(0,110) 
\figpt 9:(50,-10) \figpt 10:(0,-10)

\figpt 71:(-50,-40) \figpt 81:(50,110) 
\figpt 91:(50,-40) \figpt 101:(0,-40)

\figpt 72:(-50,-25) \figpt 82:(50,110) 
\figpt 92:(50,-25) \figpt 102:(0,-25)

\figpt 13:(-100,130) \figpt 14:(180,-10)
\figpt 15:(-50,50) \figpt 16:(80,50)

% 2. Creation of the graphical file
\figdrawbegin{}
\figdrawarrow[1,2]
\figset (color=\Redrgb)
\figdrawline[3,4]
\figdrawline[4,5]
\figset (width=default,color=default)
%\figdrawline[3,4]
%\figdrawline[4,5]
%\figset (color=default)
\figset(dash=4)
\figdrawline[4,6]
\figdrawline[41,61]
\figdrawarrow[11,12]

\figdrawend

% 3. Writing text on the figure
\figvisu{\figBoxA}{}{
\figwritec [7]{$(x_{i-1},0)$}
\figwritec [8]{$(x_i,1)=(x_j,1)$}
\figwritec [9]{$(x_{i+1},0)$}
\figwritec [10]{$(x_i,0)$}
\figwritec [71]{$(x_{j-1},0)$}
\figwritec [91]{$(x_{j+1},0)$}
\figwritec [101]{$(x_j,0)$}
\figwritec [72,92,102]{$\shortparallel$}
\figwritec [13]{$y$}
\figwritec [14]{$x$}
\figwritec [16]{$\phi_i(x)=\color{red}\phi_j(x)$}
\figset write(mark=$\figBullet$)
\figwritep[3,31,41,51,61,4,32,42,52,62,5]
}
\centerline{\box\figBoxA}
\caption{Basis functions $\phi_i(x)$ and $\phi_j(x)$. In this case, the two functions coincides}\label{basis_dia}
\end{figure}

Once again, the terms $R_i$, $i=1,\ldots,7$ will be computed separately, obtaining
\begin{align*}
	a_{i,i} = \begin{cases}
			\displaystyle h^{1-2s}\,\frac{2^{3-2s}-4}{s(1-2s)(1-s)(3-2s)}, & \displaystyle s\neq\frac{1}{2}
			\\
			\\
			8\ln 2, & \displaystyle s=\frac{1}{2}.			
			\end{cases}	
\end{align*}

\subsubsection*{Conclusion}
Summarising, we have the following values for the elements of the stiffness matrix $\mathcal{A}_h$: for $s\neq 1/2$
\begin{align*}
	a_{i,j} =  -h^{1-2s} \begin{cases}
			\displaystyle \,\frac{4(j-i+1)^{3-2s} + 4(j-i-1)^{3-2s}-6(j-i)^{3-2s}-(j-i+2)^{3-2s}-(j-i-2)^{3-2s}}{2s(1-2s)(1-s)(3-2s)}, &  \displaystyle j\geq i+2
			\\
			\\
			\displaystyle\frac{3^{3-2s}-2^{5-2s}+7}{2s(1-2s)(1-s)(3-2s)}, & \displaystyle j=i+1
			\\
			\\
			\displaystyle\frac{2^{3-2s}-4}{s(1-2s)(1-s)(3-2s)}, & \displaystyle j=i.
		\end{cases}	
\end{align*}
For $s=1/2$, instead, we have

\begin{align*}
	a_{i,j} = \begin{cases}
			-4(j-i+1)^2\log(j-i+1)-4(j-i-1)^2\log(j-i-1) 
			\\
			\;\;\;+6(j-i)^2\log(j-i)+(j-i+2)^2\log(j-i+2)+(j-i-2)^2\log(j-i-2), &  \displaystyle j> i+2
			\\
			\\
			56\ln(2)-36\ln(3), & \displaystyle j= i+2.
			\\
			\\
			\displaystyle 9\ln 3-16\ln 2, & \displaystyle j=i+1
			\\
			\\
			\displaystyle 8\ln 2, & \displaystyle j=i.
		\end{cases}	
\end{align*}





\begin{remark}
We point out the following facts:
\begin{itemize}
	\item The matrix $\mathcal A_h$ has the structure of a $N$-diagonal matrix, meaning that value of its elements remain constant along its diagonals. This is in analogy with the tridiagonal matrix approximating the classical Laplace operator. Notice, however, that in our case we obtain a full matrix. This is consistent with the non-local nature of the operator that we are discretizing.
	
	\item The value of each element $a_{i,j}$ is given explicitly, in terms only of $i$, $j$, $s$ and $h$. In other words, when approximating the left hand side of \eqref{WFD}, no numerical integration is needed. This significantly improve the efficiency of our method.
	
	\item For $s=1/2$, the elements $a_{i,j}$ do not depend on the value of $h$ which, in turn, is a function of $N$. This implies that, in this particular case, no matter how many points we consider in our mesh, the matrix $\mathcal A_h$ will always have the same entries. 
\end{itemize}
\end{remark}

\subsection{Parabolic problem and control problem}\label{fe_parabolic}

We now present the numerical implementation of the control problem for the fractional heat equation. Since we have obtained a FE approximation $\mathcal A_h$ of the operator $\fl{s}{}$, we can compute the fully-discrete approximation of \eqref{heat_frac}. 

More precisely, for any given mesh $\mathfrak M$ and any integer $M>0$, we set $\delta t=T/M$ and we consider a implicit Euler method, with respect to the time variable. More precisely, we consider
%
\begin{align}\label{frac_heat_num}
	\begin{cases}
		\displaystyle\mathcal M_h \frac{z^{n+1}-z^n}{\delta t}+\mathcal A_h z^{n+1}=\mathbf{1}_\omega v_h^{n+1}, \quad \forall n\in \left\{1,\ldots,M-1\right\}
		\\
		z^0=z_0, 
	\end{cases}
\end{align}
%
where $z_0\in \mathbb R^\mesh$ and $\mathcal M_h$ is the classical mass matrix with entries $m_{i,j}=\langle \phi_i,\phi_j\rangle$.

Here, $v_{h,\delta t}=(v_h^n)_{1\leq n\leq M}$ is a fully-discrete control function whose cost, that is the discrete $L_{\delta t}^2(0,T;\mathbb R^\mesh)$-norm, is defined by
%
\begin{align*}
\|v_{\delta t}\|_{L_{\delta t}^2(0,T;\mathbb R^\mesh)}:=\left(\sum_{i=1}^M\delta t |v^n|^2_{L^2(\Omega)}\right)^{1/2},
\end{align*}
%
and where $| \cdot |_{L^2(\Omega)}$ stands for the norm associated to the $L^2$-inner product on $\mathbb{R}^\mathfrak M$
%
\begin{align*}
(u,v)_{L^2(\Omega)}=h \sum_{i=1}^N u_i v_i.
\end{align*}
%

With the above notation and according to the penalised HUM strategy, we introduce, for some penalization parameter $\varepsilon>0$, the following primal fully-discrete functional 
%
\begin{align*}
F_{\varepsilon,h,\delta t}(v_{\delta t})=\sum_{n=1}^{M}\delta t |v^n|^2_{L^2(\Omega)}+\frac{1}{2\varepsilon}|z^M|^2_{L^2(\Omega)}, \quad \forall \, v_{\delta t}\in L_{\delta t}^2(0,T;\mathbb R^\mesh),
\end{align*}
%
that we wish to minimize onto the whole fully-discrete control space $L_{\delta t}^2(0,T;\mathbb R^\mesh)$ and where $z^M$ is the final value of the controlled problem \eqref{frac_heat_num}. 

As noted in section \rouge{numero}, we can apply Fenchel-Rockafellar theory results to obtain the corresponding dual functional, which reads as follows
%
\begin{align}\label{dual_fully}
	J_{\varepsilon,h,\delta t}(\varphi^T)=\frac{1}{2}\sum_{n=1}^{M}\delta t|\mathbf 1_\omega\varphi|^2_{L^2(\Omega)}+\frac{\varepsilon}{2}|\varphi^T|^2_{L^2(\Omega)}+(\varphi^1,y_0)_{L^2(\Omega)}, \quad \forall \varphi^T\in L_{h,\delta t}^2(0,1)
\end{align}
%
where $\varphi=(\varphi^n)_{1\leq n\leq M+1}$ is solution to the adjoint system
%
\begin{align}\label{frac_adj_num}
	\begin{cases}
		\displaystyle\mathcal M_h \frac{\varphi^n-\varphi^{n+1}}{\delta t}+\mathcal A_h \varphi^n=0, & \forall n\in\inter{1,M}
		\\
		\varphi^{M+1}=\varphi^T. 
	\end{cases}
\end{align}
%
It can be readily verified that this functional has a unique minimizer without any additional assumption on the problem. Indeeed, by minimizing  \eqref{dual_fully}, and from duality theory, we obtain a control function $v_{\varepsilon,h,\delta t}=\left(\mathbf{1}_\omega\varphi_{\varepsilon,h,\delta t}^n\right)_{1\leq n\leq M}$ where $\varphi_\epsilon$ is the solution to \eqref{frac_adj_num} evaluated in the optimal datum $\varphi_\varepsilon^T$. 

Thus, the optimal penalized control always exists and is unique, and deducing controllability properties for the systems amounts to study the behavior of this control with respect to the penalization parameter $\varepsilon$, in connection with the discretization parameters.  

It is well known that, in general, we cannot expect for a given bounded family of initial data that the fully-discrete controls are uniformly bounded when the discretization parameters $h$, $\delta t$ and the penalization term $\varepsilon$ tend to zero independently, see, for instance, \cite{}. 

Instead, we expect to obtain uniform bounds by taking the penalization parameter $\varepsilon=\phi(h)$ that tends to 0 in connection with the mesh size not too fast (see \cite{}) and a time step $\delta t$ verifying some weak condition of the kind $\delta t\leq \zeta(h)$ where $\zeta$ tends to zero logaritmically when $h\to 0$ (see \cite{}).
%