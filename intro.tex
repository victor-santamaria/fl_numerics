\section{Introduction and main results}\label{intro_sec}

In this work, we present a finite element (FE) scheme for the numerical approximation of the solution to the following non-local Poisson equation
\begin{align}\label{PE}
	\left\{\begin{array}{ll}
		\fl{s}{u} = f, & x\in(-L,L)
		\\
		u\equiv 0, & x\in\RR\setminus(-L,L).
	\end{array}\right.
\end{align}

In \eqref{PE}, $f$ is a given function and, for all $s\in(0,1)$, $\fl{s}{}$ denotes the one-dimensional fractional Laplace operator, which is defined as the following singular integral
\begin{align*}
	\fl{s}{u}(x) = \ccs\,P.V.\,\int_{\RR}\frac{u(x)-u(y)}{|x-y|^{1+2s}}\,dy. 
\end{align*}
Here, $\ccs$ is a normalisation constant given by
\begin{align*}
	\ccs = \frac{s2^{2s}\Gamma\left(\frac{1+2s}{2}\right)}{\sqrt{\pi}\Gamma(1-s)},
\end{align*}
where $\Gamma$ is the usual Gamma function. 

We have to mention that, for having a completely rigorous definition of the fractional Laplace operator, it is necessary to introduce also the class of functions $u$ for which computing $\fl{s}{u}$ makes sense. We postpone this discussion to the next section.

The analysis of non-local operators and non-local PDEs is a topic in continuous development.
A motivation for this growing interest relies in the large number of possible applications in the modeling of several complex phenomena for which a local approach turns up to be inappropriate or limiting.
Indeed, there is an ample spectrum of situations in which a non-local equation gives a
significantly better description than a PDE of the problem one wants to analyze.
Among others, we mention applications in turbulence (\cite{bakunin2008turbulence}), anomalous transport and diffusion (\cite{bologna2000anomalous,meerschaert2012fractional}), elasticity (\cite{dipierro2015dislocation}), image processing (\cite{gilboa2008nonlocal}), porous media flow (\cite{vazquez2012nonlinear}), wave propagation in heterogeneous high contrast media (\cite{zhu2014modeling}). Also, it is well known that the fractional Laplacian is the generator of s-stable processes, and it is often used in stochastic models with applications, for instance, in mathematical finance (\cite{levendorskii2004pricing,pham1997optimal}).

One of the main differences between these non-local models and classical Partial Differential Equations is that the fulfilment of a non-local equation at a point involves the values of the function far away from that point.

In the recent past, the fractional Laplacian has been widely analyzed also from the point of view of numerical analysis. We refer, for instance, to the work \cite{acosta2017fractional} of Acosta and Borthagaray (see also \cite{acosta2017short}). There, the authors present a FE scheme for implementing the solution of \eqref{PE} in a bounded domain $\Omega\subset\RR^2$. In particular, they provide appropriate quadrature rules in order to solve numerically the variational formulation associated to the problem. Moreover, in \cite{acosta2017fractional} it is also developped an accurate analysis of the efficiency of the FE method, employing several existing results. The techniques of \cite{acosta2017short,acosta2017fractional} have then been applied in \cite{acosta2017finite}, combined with a convolution quadrature approach, for solving evolution equations involving the fractional Laplacian. For the sake of completeness, we also mention \cite{bonito2015numerical}, where it is presented a discretization of the so-called \textit{spectral fractional Laplacian} (see Eq. \eqref{fl_spec}) and its application to the evolutionary case \cite{bonito2017approximation}, and \cite{nochetto2015pde}, where the solution of \eqref{PE} is implemented applying the well known extension of Caffarelli and Silvestre (\cite{caffarelli2007extension}).   

Our method deals with a FE approximation in one space dimension for the fractional Poisson equation. The main novelty of our work, with respect to \cite{acosta2017short,acosta2017fractional}, relies on the fact that, since we are dealing with the one-dimensional case, we will not need any quadrature rule and each entry of the stiffness matrix can be computed explicitly. This has the great advantage of significantly reducing the computational cost of the algorithm and, therefore, our discretization method is suitable for being included in more general applications.  

An natural example is given by the numerical resolution of the following control problem: given any $T>0$, find a control function $g\in L^2((-L,L)\times(0,T))$ such that the corresponding solution to the parabolic problem 
\begin{align}\label{heat_frac}
	\left\{\begin{array}{ll}
	z_t + \fl{s}{z} = g\mathbf{1}_{\omega}, & (x,t)\in (-L,L)\times(0,T)
	\\
	z=0, & (x,t)\in[\,\RR\setminus (-L,L)\,]\times(0,T)
	\\
	z(x,0)=z_0(x), & x\in (-L,L)
	\end{array}\right.
\end{align} 
satisfies $z(x,T)=0$. 

The approach that we will employ for solving this control problem is based on the penalized Hilbert Uniqueness Method (\cite{boyer2013penalised}), which relies on some classical works of Glowinski and Lions (\cite{glowinski1995exact,glowinski2008exact}).   

This paper is organised as follows. In Section \ref{theor_sec}, we present some existing theoretical results for the problems that we are going to analyze. In particular, we give a more accurate definition of the fractional Laplace operator and we introduce the variational formulation associated to \eqref{PE} (needed for the development of the FE scheme). Concerning the parabolic problem \eqref{heat_frac}, we recall existing controllability results, which will help us in the verification of the accuracy of the numerical method. In Section \ref{fe_sec}, we describe our FE method for the elliptic equation \eqref{PE} and we present the algorithm for the penalised HUM, employed for the numerical control of \eqref{heat_frac}. In Section \ref{res_numerical} we present and comment the results of our numerical simulations. Finally, in Appendix \ref{appendix} we include the complete details for computing the stiffness matrix associated to our FE scheme.  

